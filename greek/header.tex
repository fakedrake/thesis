
\begin{titlepage}
    \begin{center}
        \vspace*{0.15cm}

        \Large
        \textbf{Extracting relational data from Wikipedia}

        \vspace{0.8cm}
        \large{by}\\
        \large
        Χρήστος Περιβολαρόπουλος


        \vspace{2.5cm}

        Μια διπλοματική διατριβη.
        %% A thesis submitted in partial fulfillment\\
        %% of the requirements for graduation from the\\
        %% Department of Physics


        \vspace{2.2cm}
        Επιβλέποντες καθηγητές:\\
        Κυριάκος Σγάρμπας, Boris Katz

        \vspace{4cm}
        \begin{center}
        % \includegraphics[scale=0.34]{UoI2.png}
        \end{center}

        \vspace{0.6cm}
        \Large
        University of Patras

        \vspace{0.6cm}
        June 2015
    \end{center}
\end{titlepage}
\clearpage

\thispagestyle{plain}
\vspace*{1in}
\begin{center}
\begin{Large}
\textbf{\title}
\end{Large}

\vspace{0.5cm}
Χρήστος Περιβολαρόπουλος

\vspace{1.6cm}
\begin{large}
\textbf{Abstract}
\end{large}
\end{center}

Το START (SynTactic Analysis using Reversible Transformations) είναι
ένα πακέτο λογισμικού γραμμένο σε γλώσσα Common Lisp το οποίο αντλεί
πληροφορίες από διαδικτυακές πηγές και τις χρησιμοποιεί για
απαντήσει σε αυθαίρετες ερωτήσεις που δέχεται. Αναπτύχθηκε στο
εργαστήριο Infolab του MIT. Για τον εμπλουτισμό των πληροφοριών που
χρησιμοποιεί το START χρησιμοποιείται το πακέτο λογισμικού Omnibase
μέσω του οποίου επιτυγχάνεται πρόσβαση του START σε πολλαπλές πήγες
στο διαδίκτυο. Στην παρούσα εργασία αναπτύσσουμε μια επέκταση του
Omnibase που επιτρέπει την πρόσβαση του START στην wikipedia και
στις πληροφορίες που αυτή περιέχει. Η επέκταση αυτή του Omnibase
ονομάζεται wikipediabase.  Για την ευκολότερη πρόσβαση του START στη
Wikipedia δημιουργήσαμε ένα πρόγραμμα (wikipedia-mirror) που
δημιουργεί κλώνους της Wikipedia που αποθηκεύονται τοπικά ανεξάρτητα
από το διαδίκτυο.  Έτσι επιτυγχάνεται ταχύτερη και πιο αξιόπιστη
πρόσβαση του START στο σύνολο των δεδομένων της wikipedia.

START (SynTactic Analysis using Reversible Transformations) is a
piece of sofware written in common lisp that retrieves information
from internet resources and uses them to answer to arbitrary natural
language questions. It was developed in InfoLab of MiT. For the
enrichment of the retrieved information it uses the Omnibase
software through which START gets access to multiple sources on the
internet. In the present thesis we present an extension to Omnibase
that allows START to get access to wikipedia and the information
that it contains. This extension to Omnibase is called
WikipediaBase. For easier access to wikipedia we also developed a
separate program (wikipedia-mirror) that creates clones of wikipedia
running locally and independently to the internet. This way faster
and more reliable access to wikipedia is accomplished.

\vspace{3cm}
\begin{flushleft}
\textit{``The Initial Mystery that attends any journey is how did the traveler reach his starting point in the first place?''}
\end{flushleft}
\begin{flushright}
- Louise Bogan, Journey Around My Room
\end{flushright}
\afterpage{\blankpage}
\newpage


\thispagestyle{plain}
\vspace*{1in}
\begin{center}
\begin{Large}
\textit{Acknowledgements}
\end{Large}
\end{center}

First and foremost, I have to thank my research supervisor Professor L. Perivolaropoulos. Without his assistance and dedicated involvement in every step of the way, this thesis would have never been accomplished.

I also take this opportunity to express my gratitude to all the Department faculty members, for their help and support throughout my years of studying.

Lastly, I offer my sincere thanks to one and all who directly and indirectly have lent a helping hand in this virtue.
\afterpage{\blankpage}
\newpage
\tableofcontents
